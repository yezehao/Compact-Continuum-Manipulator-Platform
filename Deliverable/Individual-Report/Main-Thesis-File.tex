\documentclass[11pt]{report} % 11 pt front size
\linespread{1.5} % line spacing is 1.5
\usepackage[a4 paper, margin=1in]{geometry} % page dimensions
\headheight= 15pt
\usepackage[utf8]{inputenc} %font used is Times new roman which is the standard font for articles

\usepackage{lastpage}
\usepackage{longtable}
\usepackage{gensymb}
\usepackage{textcomp}
\usepackage[colorlinks=true,linkcolor=black,citecolor=black,urlcolor=black]{hyperref}

\usepackage[style=authoryear, style=authoryear, backend=biber, sorting=none]{biblatex} % Load biblatex
\renewcommand*{\nameyeardelim}{\addcomma\space}
\addbibresource{references.bib} 

\usepackage{fancyhdr} %for header and footer
\pagestyle{fancy}
\fancyhf{}
\usepackage{float}
\usepackage{subcaption}
\usepackage{color}
\usepackage{setspace}
\setstretch{1.5} %linespacing can be changed as per requirement
\usepackage{import}
\usepackage{titlesec}
\usepackage[export]{adjustbox}
\usepackage{tocloft}
\usepackage{supertabular}
\renewcommand\cftchapaftersnum{.}
\renewcommand\cftsecaftersnum{.}
\renewcommand\thechapter{\Roman{chapter}}
\renewcommand\thesection{\arabic{section}}
\setcounter{secnumdepth}{3} %shows detailed contents with higher sub divisions (3)
\setcounter{tocdepth}{3}
\usepackage{amsmath}
\allowdisplaybreaks
\usepackage{graphicx} %allows the user to use the graphics env
\graphicspath{{./Images/}} % The folder where the images will be uploaded
\usepackage{caption} %Allows the user to use the caption in tables and figures
\usepackage[labelfont=bf]{caption}
% \captionsetup[figure]{labelsep=space,singlelinecheck=off} % The captions won`t be affected with change in line spacing and alignment of the entire document 
\usepackage[normalem]{ulem}
\useunder{\uline}{\ul}{}
\tolerance=1
\emergencystretch=\maxdimen
\hyphenpenalty=10000
\hbadness=10000
\fancyhead[R]{8$^{th}$ Mar. 2024}
\fancyhead[L]{MECH0064 Individual Report}
\fancyfoot[R]{\thepage/\pageref{LastPage}}
\fancyfoot[L]{UCL, Department of Mechanical Engineering}
\renewcommand{\headrulewidth}{2pt} % the horizontal line at the top of the page
\renewcommand{\footrulewidth}{1pt} % the horizontal line at the bottom of the page


\begin{document}
\begin{center}
        \Large %use HUGE for very big and bold title
        \textbf{\setstretch{1.0} MECH0064 MSc Group Design Project 23/24 \\}
        \huge
        \textbf{Individual Report}
\end{center}

% =========== Start of Content page ============= %
% \tableofcontents
% \addtocontents{toc}{~\hfill\textbf{Page}\par}
% \thispagestyle{empty}
% \newpage 
% ============ End of Content page ============== %



%%%%% Section of Question 1 %%%%%
\vspace{-8mm}
\section*{Part A. Team Collaborative Effectiveness} 
\vspace{-3mm}
Although there have been some challenges in collaboration within the 
team during certain time periods, the team as a whole effectively 
engaged in cooperative efforts, which will be elucidated through 
two illustrative examples. \\
Firstly, we executed effective time management and work package 
distribution for the project, establishing a corresponding workflow. 
In terms of time management, we initiated the project by creating 
Gantt chart based on the project's objectives and deadlines. 
The Gantt chart served as a tool for project time management and 
work package distribution. As the project progressed, the timeline 
on the Gantt chart was maintained monthly and the team adheres to 
the time schedule outlined in the Gantt chart to advance the project.
Meanwhile, the work package distribution allowed all team members 
to engage in tasks aligned with their strengths, promoting efficient 
completion of work packages. Additionally, a dedicated GitHub 
repository was created upon project inception. In the README file 
of GitHub repository, we uploaded the Gantt chart and project 
milestones, utilizing various GitHub features to establish a 
workflow that supported the progression of project management 
activities. \\
Moreover, the team maintains bi-weekly meetings with supervisor to 
provide progress reports and updates the completed tasks to the 
GitHub repository. This practice facilitates intra-team awareness 
of the progress across different work packages and enhances 
collaboration when specific work packages require joint efforts. 
For instance, a team member responsible for the strain analysis of 
the manipulator need to examine the posture of manipulator while 
simulation. In such cases, the team member does not need to ask 
other team members to generate results. Instead, he can refer to 
the corresponding operational guidelines to independently obtain 
the necessary results. The documentation of each work package 
outlining programme operations contributes to the overall 
efficiency of the team. Furthermore, comprehensive documentation 
ensures that the open-source project is more accessible to a wider 
audience, facilitating ease of use for others and providing 
convenience for future team members who may continue the project.

% \begin{center}
%     \begin{longtable}{|l|l|l|l|l|l|}
%     \caption{The Maintainability of Feasible Schemes for five substations in LC.} 
%     \label{tab:maintainability}\\
%     \hline \multicolumn{1}{|c|}{\textbf{Schemes}} & 
%     \multicolumn{1}{c|}{Ref} & 
%     \multicolumn{1}{c|}{BAABB} & 
%     \multicolumn{1}{c|}{BABBB} & 
%     \multicolumn{1}{c|}{BBABB} &    
%     \multicolumn{1}{c|}{BBBBB} \\ \hline 
%     \endfirsthead
%     \multicolumn{6}{c}%
%     {{\bfseries \tablename\ \thetable{} -- continued from previous page}} \\
%     \hline \multicolumn{1}{|c|}{\textbf{Schemes}} & 
%     \multicolumn{1}{c|}{Ref} & 
%     \multicolumn{1}{c|}{BAABB} & 
%     \multicolumn{1}{c|}{BABBB} & 
%     \multicolumn{1}{c|}{BBABB} &    
%     \multicolumn{1}{c|}{BBBBB} \\ \hline 
%     \endhead
%     \hline \multicolumn{6}{|r|}{{Continued on next page}} \\ \hline
%     \endfoot
%     \hline \hline
%     \endlastfoot
%     % table context
%     \textbf{Maintenance Time}(hrs) &  & 6363 & 5629 & 6338 & 5602 \\
%     \textbf{Maintainability}(\%) & 6.06 & 1.99 & 1.76 & 1.98 & 1.75 \\
%     \textbf{Preference4}(\%) & 0 & 67.16 & 70.96 & 67.33 & 71.12 \\
%     \end{longtable}
% \end{center}


%%%%% Section of Question 2 %%%%%
\section*{Part B. Issue and Solution in Team Collaboration} 
\vspace{-3mm}
An issue faced during the group project revolves around the necessity of using GitHub as the underlying infrastructure of project, 
given its open-source nature. However, the majority of team members lack formal experience with GitHub, having only utilized 
fundamental features such as `git clone'. The difficulties arose in uploading finalized content from the local repository to the 
remote repository, which is a relative challenge for those without prior knowledge in this area. \\
In addressing this particular issue within the project, a straightforward solution was implemented. I installed git command, 
PowerShell, and other ancillary applications for each team member. After installation, comprehensive guidance was provided to 
team members on utilizing PowerShell to upload local repository files to the corresponding remote repository, and practical 
exercises were conducted to enhance their proficiency in the process. \\
The issue has been successfully addressed at the beginning of our project. However, it has come to my attention that despite my 
assistance in installing Git for team members, they have not embraced the practice of utilizing GitHub while doing the project. 
Consequently, challenges have arisen in consistently synchronizing the progress of work packages and adhering to the established 
workflow. Additionally, there has been a notable difficulty in monitoring individual contributions and tracing the developmental 
trajectory of the project. It also proved difficult to track individual contributions and monitor the developmental trajectory of 
the project. These aspects are crucial in the development of an open-source project, as the project's evolution is an integral part 
of its overall progress. Simply uploading completed content into the GitHub repository falls short of the expected practices in a 
well-governed open-source project. Timely synchronization of each step's content and comprehensive project documentations are 
imperative elements that should be consistently accomplished. \\
The experience gained during the resolution of this issue indicates that, before involving team members in a workflow, the leader 
should clarify the purpose of adopting the workflow, rather than solely instructing them on how to use the tools within the workflow. 
The collaborative mindset of a team equipped with a workflow is of paramount importance. By clearly defining the objectives, team 
members can engage more meaningfully with the workflow, ultimately optimizing efficiency. Furthermore, during the initial phases of 
project, there are minimal intensive tasks. The simulated assignments can be introduced as a mock test to assist team members being 
more familiar with the workflow. This method effectively equips members to interact with the workflow more adeptly, leading to heightened 
overall efficiency.



%%%%% REFERENCES %%%%%
% \newpage
% \pagenumbering{roman}
% \addcontentsline{toc}{chapter}{References}
% \printbibliography[title={References}]
% \pagestyle{plain}

\end{document}
%------------------------------------------The Document Ends here--------------------------------------------