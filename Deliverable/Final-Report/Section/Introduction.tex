%%%%% Introduction %%%%%
\section{Introduction} 
\subsection{Background}
Robotic manipulators refer to devices that help operators perform repetitive tasks without direct physical contact at a very 
fast speed [1]. In the early stages of robotic manipulator development, rigid joint manipulators dominated the field and were 
mainly used in industrial production to perform simple tasks previously done manually, or being applied in hazardous environments 
to avoid human casualties. As this technology progressed, various types of manipulators were developed and utilized in different 
fields. In specific fields such as healthcare [2], precision manufacturing, and searching and rescuing, there are strict 
requirements for the precision, motion trajectory, and morphology of robotic arms. Thus, continuum manipulators emerged. 
Their flexible structure allows them to attain any desired posture to reach target positions, making them suitable to operate 
in complex environments [3]. In recent years, continuum manipulators have been widely applied, saving time and labor costs while 
reducing potential risks brought about by inaccuracy. 

\subsection{Motivation}
Continuum robots have wide applications in the medical field, especially in surgeries requiring extremely high precision and 
specific postures to complete tasks. They are smaller, more flexible, and can increase the success rate of surgeries while 
minimizing patient discomfort. 

Among the various surgical operations that continuum manipulators can perform, there is one certain surgery that requires the 
application of a Focused Ultrasound (FUS) transducer to emit ultrasound waves to penetrate the human skin, then bombard and 
destroy tumor tissues buried beneath the skin [4]. In the past, manipulators capable of performing this surgery existed, but 
they were very expensive [5] and not easy for doctors to operate.

\subsection{Objective}
The objective of the project aims to design a modular fishbone continuum manipulator constructed from commonly used materials, 
capable of carrying the FUS transducer. The workspace of manipulator is specified as 300x300x300mm,  featuring a high level 
of precision with a permissible error margin of 0.02mm. The manipulator is expected to be user-friendly, requiring a learning 
cost of less than two hours. Testing and validation will be conducted through simulations based on Ansys, MATLAB, and Arduino. 
The project commenced on 10$^{th}$ November 2023, and is scheduled to conclude on 15$^{th}$ March 2024.


% change to new page
\newpage