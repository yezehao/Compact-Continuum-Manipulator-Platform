%%%%% Introduction %%%%%
\section{Introduction} 
\subsection{Background}
This is the background part.
\subsection{Motivation}
This is the motivation part.

\subsection{Introduction}
Continuum robots has emerged and attracted a lot of attention since 2008 \cite{review_2008}. Before that, rigid 
joint robots were dominating the robotic arm industry. Compared with traditional rigid joint manipulator, continuum 
robots stand out for their flexible, highly bendable structure and extremely flexible motion performance. The 
limitations of rigid joint robots have gradually shown up in applications requiring highly detailed operation and 
in complex or space-limited environments. Thus, the continuum robot was developed and perfected during the years. 
This new kind of robot not only changes the code of traditional robot design but also demonstrates unprecedented 
application potential in fields such as exploration industry and medical science \cite{CR_medical_application}. 
Meanwhile, rigid-flexible-soft coupled continuum robots combine the multiple advantages of the stability of rigid 
structures, the flexibility of bendable structures, and the compliance of soft structures, and are one of the most 
promising robots for increasingly complex tasks \cite{fishboneCR}. \\
With unique bionic structure and motion characteristics, continuum robots provide new possibilities to solve these 
challenges. \\
This paper will discuss different types of existing continuum robots and their working principles, advantages, 
and disadvantages, then propose a proper continuum robot design that can be mainly applied to medical applications.

% change to new page
\newpage