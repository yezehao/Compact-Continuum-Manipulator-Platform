%%%%% Introduction %%%%%
\section{Introduction} 
\subsection{Background}
Robotic manipulators refer to devices that help operators perform repetitive tasks without direct physical contact at a very 
fast speed [1]. In the early stages of robotic manipulator development, rigid joint manipulators dominated the field and were 
mainly used in industrial production to perform simple tasks previously done manually, or being applied in hazardous environments 
to avoid human casualties. As this technology progressed, various types of manipulators were developed and utilized in different 
fields. In specific fields such as healthcare [2], precision manufacturing, and searching and rescuing, there are strict 
requirements for the precision, motion trajectory, and morphology of robotic arms. Thus, continuum manipulators emerged. 
Their flexible structure allows them to attain any desired posture to reach target positions, making them suitable to operate 
in complex environments [3]. In recent years, continuum manipulators have been widely applied, saving time and labor costs while 
reducing potential risks brought about by inaccuracy. 

\subsection{Motivation}
Continuum robots have wide applications in the medical field, especially in surgeries requiring extremely high precision and 
specific postures to complete tasks. They are smaller, more flexible, and can increase the success rate of surgeries while 
minimizing patient discomfort. 

Among the various surgical operations that continuum manipulators can perform, there is one certain surgery that requires the 
application of a Focused Ultrasound (FUS) transducer to emit ultrasound waves to penetrate the human skin, then bombard and 
destroy tumor tissues buried beneath the skin [4]. In the past, manipulators capable of performing this surgery existed, but 
they were very expensive [5] and not easy for doctors to operate.

This project aims to design a continuum manipulator to manipulate FUS transducers for surgical operations on the skin area 
of patients. At the same time, it is expected to solve the challenges that previous manipulators were facing, namely:
Cost: The manipulator designed in this project needs to be manufacturable at an extremely low cost and be easy to mass-produce.
Reliability: This manipulator needs to operate with high precision, have a wide range of workspace, and be able to handle certain mass 
loads without causing any damage to itself.
Simplicity: The manipulator needs to be easy to use.

Objectives
To design a 4-unit fish-bone continuum manipulator that can be applied in FUS applications, having a workspace of 300x300x300mm. 
The test and verification are based on Ansys, Matlab, and Arduino simulations. The project starts on 10th Nov. 2023 and will 
finish on 15th Mar. 2024.


% change to new page
\newpage