%%%%% Conclusion %%%%%
\section{Conclusion} 
The project provides an introduction to the background and justifies the selection of a  fishbone continuum manipulator. 
It demonstrated simulations based on the specified objectives, with the methodology encompassing structural modeling, 
kinematic derivation, and control system development.

In this report, structural modeling was conducted using Ansys for strength and strain analysis. The conclusion drawn was that 
rubber, serving as the backbone structure, exhibited sufficient strength, and the motor torque was adequate to provide the 
corresponding tension in the cables. However, it is important to note that fatigue loss has not been taken into consideration.

From the perspective of kinematics, the corresponding workspace was determined through forward kinematics. However, a limitation 
was observed in that a strong correlation exists between the majority of positions and orientations. On the other hand, inverse 
kinematics accurately reproduced the posture but existing some degree of error. A proposed method was introduced as a potential 
solution to address this issue.

The control system was implemented to manage functions such as motor control. Simultaneously, a set of sensor systems has 
been deployed to prepare for the upgrade to a closed-loop system.

In conclusion, this project exhibited a high level of completion and relative comprehensiveness. All content, including this 
report, has been uploaded to GitHub, accompanied by corresponding README files and code comments to guide users in executing 
the programmes. The \href{https://github.com/yezehao/Compact-Continuum-Manipulator-Platform}{GitHub Repository} can be accessed 
through https://github.com/yezehao/Compact-Continuum-Manipulator-Platform .



% change to new page
\newpage