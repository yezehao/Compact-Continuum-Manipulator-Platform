\documentclass[12pt]{report} % 12 pt front size
\usepackage[a4 paper, top=25mm, bottom=25mm]{geometry} % page dimensions
\headheight= 15pt
\usepackage[utf8]{inputenc} %font used is Times new roman which is the standard font for articles

%Bibliography package
\usepackage[backend=biber,style=nature]{biblatex}
% .bib is the name of the file which contains the list of references for citation
\addbibresource{references.bib}
\usepackage{fancyhdr} %for header and footer
\pagestyle{fancy}
\fancyhf{}
\usepackage{float}
\usepackage{subcaption}
\usepackage{color}
\usepackage{setspace}
\setstretch{1.5} %linespacing can be changed as per requirement
\usepackage{import}
\usepackage{titlesec}
\usepackage[export]{adjustbox}
\usepackage{tocloft}
\usepackage{supertabular}
\renewcommand\cftchapaftersnum{.}
\renewcommand\cftsecaftersnum{.}
\renewcommand\thechapter{\Roman{chapter}}
\renewcommand\thesection{\arabic{section}}
\setcounter{secnumdepth}{3} %shows detailed contents with higher sub divisions (3)
\setcounter{tocdepth}{3}
\usepackage{amsmath}
\usepackage{graphicx} %allows the user to use the graphics env
\graphicspath{{./Images/}} % The folder where the images will be uploaded
\usepackage{caption} %Allows the user to use the caption in tables and figures
\usepackage[labelfont=bf]{caption}
\captionsetup[figure]{labelsep=space,singlelinecheck=off} % The captions won`t be affected with change in line spacing and alignment of the entire document 
\usepackage[normalem]{ulem}
\useunder{\uline}{\ul}{}
\tolerance=1
\emergencystretch=\maxdimen
\hyphenpenalty=10000
\hbadness=10000
\fancyhead[R]{20th Nov. 2023}
\fancyhead[L]{Department of Mechanical Engineering, UCL}
\fancyfoot[R]{\thepage}
\renewcommand{\headrulewidth}{2pt} % the horizontal line at the top of the page
\renewcommand{\footrulewidth}{1pt} % the horizontal line at the bottom of the page

% ==================== Start of Cover Page ==================== %
\begin{document}
\begin{titlepage}
\begin{center}
        \begin{center}
        \includegraphics[width=.70\textwidth]{Image/ucl_logo.png}
        \vspace{0.5cm}
        \end{center}
        \vspace*{0.1cm}
        % Title
        \vspace{2cm}
        {\LARGE\textbf{MECH0064 MSc Group Design Project\\}}

        % Sub-Title
        \vspace{2cm}
        {\Huge\textbf{Compact Continuum Robotic\\}}
        \vspace{0.25cm}
        {\Huge\textbf{Manipulator Platform\\}}

        \vfill

        \begin{flushleft}
        \textbf{\emph{Group members:}} \\
        Zehao Ye (23119333)~~ Zehao Ye (23119333)\\ 
        Zehao Ye (23119333)~~ Zehao Ye (23119333)\\ 
        Zehao Ye (23119333)~~ Zehao Ye (23119333)\\ 
        \textbf{\emph{Supervised by} Dr Reza Haqshenas}
        \end{flushleft}
        \vspace{0.8cm}
\end{center}
\end{titlepage}
% ==================== End of Cover Page ==================== %

%%%%% Abstract %%%%%
\section*{Abstract} 
The robotic manipulator plays a crucial role in mitigating the labor for people across different fields. This project designed a 
specialized 4-unit fishbone continuum manipulator intended for medical applications. The end effector of this manipulator is 
going to be equipped with a Focused Ultrasound (FUS) transducer, enabling the eradication of tumor tissues within the human body. 
The project methodology is divided into three parts, including modeling and strain analysis, derivation of forward and inverse 
kinematics, and actuation control strategies. 

Through this approach, a viable design solution for the continuum manipulator has been crafted. By using 8 motors to control 
the 8 cables of the manipulator separately, we achieved an 8 DOF configuration with a working space of 300x300x300mm. The 
resulting manipulator boasts user-friendly operation, affordability, and remarkable precision. At the same time, there are 
still some improvements to be made to optimize the design, namely the closed-loop feedback systems and the structural strength 
analysis. 

Looking ahead, the project should focus on the transition toward the practical manufacturing phase, where the simulation result a
nd parameters can be verified by producing a real manipulator prototype.


\vfill


\textbf{\emph{Key Words: Continuum Robotic, Therapeutic Ultrasound, Robot Kinematics, Strain Analysis, Arduino}}
\vspace{0.8cm}
\pagenumbering{gobble}
\setcounter{page}{1}
\newpage

% =========== Start of Content page ============= %
\tableofcontents
\addtocontents{toc}{~\hfill\textbf{Page}\par}
\newpage 
% ============ End of Content page ============== %


% =========== Start of Figure List ============= %
\listoffigures
\addcontentsline{toc}{chapter}{List of Figures}
\newpage
% ============ End of Figure List ============== %



% =========== Start of Table List ============= %
\listoftables 
\addcontentsline{toc}{chapter}{List of Tables} 
\newpage
% ============ End of Table List ============== %



%%%%% Section of Question 1 %%%%%
\section{Question 1. Diode bridge circuit (4\%)} 
%%%%% Subsection of Q1 a) %%%%%
\subsection{The circuit diagram from PSCAD} 
The screen capture of the circuit diagram from PSCAD is shown in Figure \ref{fig:Q1circuitdiagram}. 
\begin{figure}[H] %[H] "corresponds to start the figure Here" 
    \centering %alignment can be flushleft or flushright
    \includegraphics[width=0.9\textwidth]{Image/Q1/Q1_circuit.PNG} 
    \caption[The screen capture of your circuit diagram from PSCAD]
    {\centering \textit{\textbf{The circuit diagram:}} Screen capture about Question 1-a in PSCAD}
    \label{fig:Q1circuitdiagram}
\end{figure} %ends the image environment
%%%%% Subsection of Q1 b) %%%%%
\subsection{The instantaneous voltage measurement (5mF)} 
The screen capture of the circuit diagram from PSCAD is shown in Figure \ref{fig:Q1A2Figure}. 
\begin{figure}[H] %[H] "corresponds to start the figure Here" 
    \centering % subfigure1
    \begin{subfigure}[b]{\textwidth}
        \centering
        \includegraphics[width=0.7\textwidth]{Image/Q1/Q1_a2_Vsource.PNG}
        \caption{The instantaneous input voltage across the voltage source (unit: kV)}
        \label{fig:Q1A2sub1}
    \end{subfigure}
    \hfill
    \centering % subfigure2
    \begin{subfigure}[b]{\textwidth}
        \centering
        \includegraphics[width=0.7\textwidth]{Image/Q1/Q1_a2_Vload.PNG}
        \caption{The instantaneous output voltage across the resistive load (unit: kV)}
        \label{fig:Q1A2sub2}
    \end{subfigure} 
    \caption[The measured instantaneous voltages with 5mF capacitor]
    {\centering \textit{\textbf{The measured instantaneous voltages:}} V\textsubscript{source} and R\textsubscript{load}}
    \label{fig:Q1A2Figure}
\end{figure}
%%%%% Subsection of Q1 c) %%%%%
\subsection{The instantaneous voltage measurement (25mF)} 
%%%%% FIGURE %%%%%
\begin{figure}[H]
    \centering % subfigure1
    \begin{subfigure}[b]{\textwidth}
        \centering
        \includegraphics[width=0.9\textwidth]{Image/Q1/Q1_a3_circuit.PNG}
        \caption{The circuit diagram with different capacitor}
        \label{fig:Q1A3sub1}
    \end{subfigure}
    \hfill
    \centering % subfigure2
    \begin{subfigure}[b]{\textwidth}
        \centering
        \includegraphics[width=0.7\textwidth]{Image/Q1/Q1_a3_Vload.PNG}
        \caption{The instantaneous output voltage across the resistive load (unit: kV)}
        \label{fig:Q1A3sub2}
    \end{subfigure} 
    \caption[The measured instantaneous voltages with 25mF capacitor]
    {\centering \textit{\textbf{The measured instantaneous voltages:}} R\textsubscript{load}with 25mF}
    \label{fig:Q1A3Figure}
\end{figure}
%%%%% TEXT %%%%%
To make further analysis about the impact of different capacitors on the entire circuit, the data in 
Figure \ref{fig:Q1A2sub1}, \ref{fig:Q1A2sub2}, and \ref{fig:Q1A3sub2} are exported to MATLAB. 
The peaks of the measured instantaneous voltage are labeled and the average values are plotted in Figure \ref{fig:Q1MATLAB}.
The purpose of increasing the capacitance is to reduce circuit oscillations caused by the AC power source and reduce steady state error. 
It can be observed in Figure \ref{fig:Q1MATLAB} that with a larger capacitor, the instantaneous voltage fluctuations across 
the resistive load are smaller, leading to a comparatively more stable condition.
%%%%% FIGURE %%%%%
\begin{figure}[H]
    \centering % subfigure1
    \begin{subfigure}[b]{\textwidth}
        \centering
        \includegraphics[width=0.65\textwidth]{Image/Q1/Q1A2Vsource.png}
        \caption{The instantaneous input voltage across the voltage source in MATLAB}
        \label{fig:Q1MATLABsub1}
    \end{subfigure}
    \hfill
    \centering % subfigure2
    \begin{subfigure}[b]{\textwidth}
        \centering
        \includegraphics[width=0.65\textwidth]{Image/Q1/Q1A2Vload.png}
        \caption{The instantaneous output voltage across the resistive load (5mF) in MATLAB}
        \label{fig:Q1MATLABsub2}
    \end{subfigure} 
    \hfill
    \centering % subfigure3
    \begin{subfigure}[b]{\textwidth}
        \centering
        \includegraphics[width=0.65\textwidth]{Image/Q1/Q1A3Vload.png}
        \caption{The instantaneous output voltage across the resistive load (25mF) in MATLAB}
        \label{fig:Q1MATLABsub3}
    \end{subfigure} 
    \caption[The processed instantaneous voltage measurements in MATLAB]
    {\centering \textit{\textbf{The processed instantaneous voltages:}} Processed in MATLAB, the unit in the MATLAB plots are kiloVolt(kV)}
    \label{fig:Q1MATLAB}
\end{figure}
\newpage
%%%%%%%%%%%%%%%%%%%%%%%%%%%%%%%%%%%% NEW SECTION %%%%%%%%%%%%%%%%%%%%%%%%%%%%%%%%%%%%%%%%


%%%%% Section of Question 2 %%%%%
\section{Question 2. Equivalent transformer (4\%)} 
%%%%% Subsection of Q2 a) %%%%%
\subsection{The circuit diagram from PSCAD}
The screen capture of the circuit diagram from PSCAD about the resistive, inductive, and capacitive laod 
are shown in Figure \ref{fig:Q2circuit_sub1}, \ref{fig:Q2circuit_sub2}, and \ref{fig:Q2circuit_sub3} respectively. 
%%%%% FIGURE %%%%%
\begin{figure}[H]
    \centering % subfigure1
    \begin{subfigure}[b]{\textwidth}
        \centering
        \includegraphics[width=0.9\textwidth]{Image/Q2/Q2_resistor_circuit.PNG}
        \caption{The circuit diagram with the resistive load}
        \label{fig:Q2circuit_sub1}
    \end{subfigure}
    \hfill
    \centering % subfigure2
    \begin{subfigure}[b]{\textwidth}
        \centering
        \includegraphics[width=0.9\textwidth]{Image/Q2/Q2_inductor_circuit.PNG}
        \caption{The circuit diagram with the inductive load}
        \label{fig:Q2circuit_sub2}
    \end{subfigure} 
    \hfill
    \centering % subfigure3
    \begin{subfigure}[b]{\textwidth}
        \centering
        \includegraphics[width=0.9\textwidth]{Image/Q2/Q2_capacitor_circuit.PNG}
        \caption{The circuit diagram with the capacitive load}
        \label{fig:Q2circuit_sub3}
    \end{subfigure} 
    \caption[The circuit diagrams of the different loads]
    {\centering \textit{\textbf{The circuit diagrams:}} resistive, inductive, and capacitive load}
    \label{fig:Q2circuitdiagram}
\end{figure}
\newpage
%%%%% Subsection of Q2 b) %%%%%
\subsection{The measurement across the resistive load}
The RMS voltage (analogue) and power factor (angular) about the resistive laod are shown in Figure \ref{fig:Q2R_RMS} and \ref{fig:Q2R_PF}. 
%%%%% FIGURE %%%%%
\begin{figure}[H]
    \centering % subfigure1
    \begin{subfigure}[b]{\textwidth}
        \centering
        \includegraphics[width=0.9\textwidth]{Image/Q2/Q2_resistor_RMS.PNG}
        \caption{The RMS voltage about the resistive load (unit: kV)}
        \label{fig:Q2R_RMS}
    \end{subfigure}
    \hfill
    \centering % subfigure2
    \begin{subfigure}[b]{\textwidth}
        \centering
        \includegraphics[width=0.9\textwidth]{Image/Q2/Q2_resistor_PF.PNG}
        \caption{The power factor about the resistive load (unit: degree~$^\circ$)}
        \label{fig:Q2R_PF}
    \end{subfigure} 
    \caption[The RMS voltage and power factor about the resistive load ]
    {\centering \textit{\textbf{The measurements about the resistive load:}} RMS Voltage(analogue) and Power Factor(angular)}
    \label{fig:Q2R}
\end{figure}
\newpage
%%%%% Subsection of Q2 c) %%%%%
\subsection{The measurement across the inductive load}
The RMS voltage (analogue) and power factor (angular) about the inductive laod are shown in Figure \ref{fig:Q2L_RMS} and \ref{fig:Q2L_PF}. 
%%%%% FIGURE %%%%%
\begin{figure}[H]
    \centering % subfigure1
    \begin{subfigure}[b]{\textwidth}
        \centering
        \includegraphics[width=0.9\textwidth]{Image/Q2/Q2_inductor_RMS.PNG}
        \caption{The RMS voltage about the inductive load (unit: kV)}
        \label{fig:Q2L_RMS}
    \end{subfigure}
    \hfill
    \centering % subfigure2
    \begin{subfigure}[b]{\textwidth}
        \centering
        \includegraphics[width=0.9\textwidth]{Image/Q2/Q2_inductor_PF.PNG}
        \caption{The power factor about the inductive load (unit: degree~$^\circ$)}
        \label{fig:Q2L_PF}
    \end{subfigure} 
    \caption[The RMS voltage and power factor about the inductive load ]
    {\centering \textit{\textbf{The measurements about the inductive load:}} RMS Voltage(analogue) and Power Factor(angular)}
    \label{fig:Q2L}
\end{figure}
\newpage
%%%%% Subsection of Q2 d) %%%%%
\subsection{The measurement across the capacitive load}
The RMS voltage (analogue) and power factor (angular) about the capacitive laod are shown in Figure \ref{fig:Q2C_RMS} and \ref{fig:Q2C_PF}. 
%%%%% FIGURE %%%%%
\begin{figure}[H]
    \centering % subfigure1
    \begin{subfigure}[b]{\textwidth}
        \centering
        \includegraphics[width=0.9\textwidth]{Image/Q2/Q2_capacitor_RMS.PNG}
        \caption{The RMS voltage about the capacitive load (unit: kV)}
        \label{fig:Q2C_RMS}
    \end{subfigure}
    \hfill
    \centering % subfigure2
    \begin{subfigure}[b]{\textwidth}
        \centering
        \includegraphics[width=0.9\textwidth]{Image/Q2/Q2_capacitor_PF.PNG}
        \caption{The power factor about the capacitive load (unit: degree~$^\circ$)}
        \label{fig:Q2C_PF}
    \end{subfigure} 
    \caption[The RMS voltage and power factor about the capacitive load ]
    {\centering \textit{\textbf{The measurements about the capacitive load:}} RMS Voltage(analogue) and Power Factor(angular)}
    \label{fig:Q2C}
\end{figure}
\newpage
%%%%%%%%%%%%%%%%%%%%%%%%%%%%%%%%%%%% NEW SECTION %%%%%%%%%%%%%%%%%%%%%%%%%%%%%%%%%%%%%%%%



%%%%% Section of Question 3 %%%%%
\section{Question 3. Faulted 3-$\Phi$ network PART 1 (4\%)} 
\newpage
%%%%%%%%%%%%%%%%%%%%%%%%%%%%%%%%%%%% NEW SECTION %%%%%%%%%%%%%%%%%%%%%%%%%%%%%%%%%%%%%%%%



%%%%% Section of Question 4 %%%%%
\section{Question 4. Faulted 3-$\Phi$ network PART 2 (5\%)} 
\newpage
%%%%%%%%%%%%%%%%%%%%%%%%%%%%%%%%%%%% NEW SECTION %%%%%%%%%%%%%%%%%%%%%%%%%%%%%%%%%%%%%%%%



%%%%% Section of Question 5 %%%%%
\section{Question 5. Faulted 3-$\Phi$ network PART 3 (8\%)} 
\newpage
%%%%%%%%%%%%%%%%%%%%%%%%%%%%%%%%%%%% NEW SECTION %%%%%%%%%%%%%%%%%%%%%%%%%%%%%%%%%%%%%%%%



%%%%% Section of Question 6 %%%%%
\section{Question 6. . Faulted Network Analysis (10\%)} 
\newpage
%%%%%%%%%%%%%%%%%%%%%%%%%%%%%%%%%%%% NEW SECTION %%%%%%%%%%%%%%%%%%%%%%%%%%%%%%%%%%%%%%%%

\printbibliography[title = {References}]
\addcontentsline{toc}{chapter}{References}
\end{document}
%------------------------------------------The Document Ends here--------------------------------------------